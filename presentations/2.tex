\documentclass{beamer}

\usepackage[utf8]{inputenc}
\usepackage{hyperref}
\usetheme{metropolis}

% Aliasing
\let\tt\texttt
\let\bf\textbf
\let\it\itshape
\let\ul\underline

\title{Using the shell}
\date{3 November 2021}
\author{Johan Öhlin}
\institute{Yabs}

\begin{document}
% Page 1 (title) %%%%%%%%%%%%%%%%%%%%%%%%%%%%%%%%%%%%%%%%%%%%%%%%%%%%%%%%%%%%%
\begin{frame}{Using the shell}
\maketitle
\end{frame}

% Page 2 %%%%%%%%%%%%%%%%%%%%%%%%%%%%%%%%%%%%%%%%%%%%%%%%%%%%%%%%%%%%%%%%%%%%%
\begin{frame}{grep}
        \tt{\$ whatis grep}\\
        \tt{grep (1)  - print lines that match patterns}\\
        \tt{\$ cat file.txt} \\
        \tt{word} \\
        \tt{hello} \\
        \tt{swordfish} \\
        \tt{\$ grep -n word     file.txt} \\
        \tt{1:word} \\
        \tt{3:swordfish} \\
        \tt{\$ grep -ci WORD    file.txt} \\
        \tt{2} \\
        \tt{\$ } \\
\end{frame}

% Page 3 %%%%%%%%%%%%%%%%%%%%%%%%%%%%%%%%%%%%%%%%%%%%%%%%%%%%%%%%%%%%%%%%%%%%%
\begin{frame}{grep}
        Find exercises from GitHub: \\
        \url{https://github.com/Najoj/yabsbashcourse/blob/main/exercises/2/grep}
\end{frame}

% Page 4 %%%%%%%%%%%%%%%%%%%%%%%%%%%%%%%%%%%%%%%%%%%%%%%%%%%%%%%%%%%%%%%%%%%%%
\begin{frame}{sed}
        \tt{\$ whatis sed}\\
        \tt{sed (1)   - stream editor for filtering and transforming text}\\
        \tt{\$ cat file.txt} \\
        \tt{word} \\
        \tt{hello} \\
        \tt{swordfish} \\
        \tt{\$ sed "s/ord/in/" file.txt} \\
        \tt{win} \\
        \tt{hello} \\
        \tt{swinfish} \\
        \tt{\$ } \\
\end{frame}

% Page 5 %%%%%%%%%%%%%%%%%%%%%%%%%%%%%%%%%%%%%%%%%%%%%%%%%%%%%%%%%%%%%%%%%%%%%
\begin{frame}{sed}
        \tt{\$ sed "2,4d"              file.txt}\\
        \tt{\$ sed "s/this/that/"      file.txt}\\  
        \tt{\$ sed "s/this/that/g"     file.txt}\\  
        \tt{\$ sed '10,20s/this/that/' file.txt}\\  
        \tt{\$ sed '10,20d'            file.txt}\\  
        \tt{\$ sed 'p'                 file.txt}\\  
        \tt{\$ sed -e '2p' -e '3d"     file.txt}\\  
        \tt{\$ sed -i "s/this/that/g"  file.txt}\\  
        \tt{\$ sed -f script-file.sed  file.txt}\\  
\end{frame}

% Page 6 %%%%%%%%%%%%%%%%%%%%%%%%%%%%%%%%%%%%%%%%%%%%%%%%%%%%%%%%%%%%%%%%%%%%%
\begin{frame}{sed}
        Find exercises from GitHub: \\
        \url{https://github.com/Najoj/yabsbashcourse/blob/main/exercises/2/sed}
\end{frame}

% Page 7 %%%%%%%%%%%%%%%%%%%%%%%%%%%%%%%%%%%%%%%%%%%%%%%%%%%%%%%%%%%%%%%%%%%%%
\begin{frame}{awk}
        \tt{\$ whatis awk}\\
        \tt{awk (1)   - pattern scanning and text processing language}\\
        \tt{\$ awk "\{ print \$1 \}" } \\
        \tt{\$ awk 'BEGIN \{ printf "wc -l{\textbackslash}n" \} } \\
        \tt{\quad \quad \{ s+=1 \}} \\
        \tt{\quad END \{ printf "\%d{\textbackslash}n", s \}' file.txt } \\
        \tt{\$ awk "/match/ \{\ldots \} < file.txt } \\
        \tt{\$ awk -f script-file.awk file1.txt file2.txt } \\
\end{frame}

% Page 8 %%%%%%%%%%%%%%%%%%%%%%%%%%%%%%%%%%%%%%%%%%%%%%%%%%%%%%%%%%%%%%%%%%%%%
\begin{frame}{awk}
        Built-in variables and functions\\
        \begin{tabular}{l|l}
                \hline
            ARGC     & Number of arguments. \\
            NR       & Lines read so far.   \\
            ENVIRON  & Environment variables. \\
            FILENAME & Name of file provided. \\
                \hline
            getline  & Set \$0 to next input line. \\
            next     & Skip to next line. \\
            nextfile & Skip the rest of this file. \\
                \hline
        \end{tabular} \\
        \medskip
        See \tt{man} page for more. \\
        \tt{\$ sed 'END \{ print NR \}' < file.txt} \\
\end{frame}

% Page 9 %%%%%%%%%%%%%%%%%%%%%%%%%%%%%%%%%%%%%%%%%%%%%%%%%%%%%%%%%%%%%%%%%%%%%
\begin{frame}{awk}
        Find exercises from GitHub: \\
        \url{https://github.com/Najoj/yabsbashcourse/blob/main/exercises/2/awk}
\end{frame}

% Page 10 %%%%%%%%%%%%%%%%%%%%%%%%%%%%%%%%%%%%%%%%%%%%%%%%%%%%%%%%%%%%%%%%%%%%
\begin{frame}{find}
        Does {\it not} take files as arguments. \\
        \tt{\$ whatis find}\\
        \tt{find (1)  - search for files in a directory hierarchy}\\
        \tt{\$ find /home/user -name "file.txt"} \\
        \tt{./file.txt} \\
        \tt{\$ find /home -name "file.txt" -exec grep "hi" \{\} + } \\
        \tt{\$ find / -name "file.txt" -exec grep "hi" \{\} {\textbackslash}\\; } \\
        \tt{\$ } \\
\end{frame}

% Page 11 %%%%%%%%%%%%%%%%%%%%%%%%%%%%%%%%%%%%%%%%%%%%%%%%%%%%%%%%%%%%%%%%%%%%
\begin{frame}{find}
        Useful options:\\
        \begin{tabular}{| l | l}
                \hline
                \tt{-name n}            &   Name of file, case sensitive \\
                \tt{-iname n}           &   Name of file, case insensitive \\
                \tt{-path n}            &   Path of file, case sensitive \\
                \tt{-ipath n}           &   Path of file, case insensitive \\
                \tt{-type n}            &   Type of file. \tt{d} directory, \tt{f} file, etc. \\
                \tt{-maxdepth n}        &   Descend at most \tt{n} directories \\
                \tt{-mindepth n}        &   Descend at least \tt{n} directories \\
                \tt{-exec cmd \{\} + }  &   Execute \tt{cmd} on each file found \\
                \hline
        \end{tabular} \\
        \medskip
        Can use some logic, for example: \\
        \tt{find /some/dir -name n -or -name m}
\end{frame}

% Page 12 %%%%%%%%%%%%%%%%%%%%%%%%%%%%%%%%%%%%%%%%%%%%%%%%%%%%%%%%%%%%%%%%%%%%
\begin{frame}{find}
        Find exercises from GitHub: \\
        \url{https://github.com/Najoj/yabsbashcourse/blob/main/exercises/2/find}
\end{frame}

% Page 13 %%%%%%%%%%%%%%%%%%%%%%%%%%%%%%%%%%%%%%%%%%%%%%%%%%%%%%%%%%%%%%%%%%%%
\begin{frame}{xargs}
        \tt{\$ whatis xargs}\\
        \tt{xargs (1)  - build and execute command lines from standard input}\\
        \tt{\$ cat file.txt} \\
        \tt{a} \\
        \tt{b} \\
        \tt{fileB.txt} \\
        \tt{\$ cat file.txt | xargs ls} \\
        \tt{a:} \\
        \tt{fileA.txt} \\
        \tt{} \\
        \tt{b:} \\
        \tt{fileB.txt} \\
        \tt{\$ ls a b} \\
\end{frame}

% Page 14 %%%%%%%%%%%%%%%%%%%%%%%%%%%%%%%%%%%%%%%%%%%%%%%%%%%%%%%%%%%%%%%%%%%%
\begin{frame}{xargs}
        Find exercises from GitHub: \\
        \url{https://github.com/Najoj/yabsbashcourse/blob/main/exercises/2/xargs}
\end{frame}

% Page 15 %%%%%%%%%%%%%%%%%%%%%%%%%%%%%%%%%%%%%%%%%%%%%%%%%%%%%%%%%%%%%%%%%%%%
\begin{frame}{Summary}
        Use the right tool for the right task. \\
        \tt{sed} and \tt{awk} are languages themselves, we have only touched upon them. \\
        \tt{grep}, \tt{sed}, and \tt{awk} can usually execute the same task. \\
        \tt{find} is a multi-tool; it can do a lot at once. \\
        \tt{xargs} does one thing well.
\end{frame}
% Page 15 %%%%%%%%%%%%%%%%%%%%%%%%%%%%%%%%%%%%%%%%%%%%%%%%%%%%%%%%%%%%%%%%%%%%
\begin{frame}{Summary}
        Use the right tool for the right task. \\
        \tt{sed} and \tt{awk} are languages themselves, we have only touched upon them. \\
        \tt{grep}, \tt{sed}, and \tt{awk} can usually execute the same task. \\
        \tt{find} is a multi-tool; it can do a lot at once. \\
        \tt{xargs} does one thing well.
\end{frame}
\end{document}
