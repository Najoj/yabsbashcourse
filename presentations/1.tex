\documentclass{beamer}

\usepackage[utf8]{inputenc}
\usepackage{hyperref}
\usetheme{metropolis}

% Aliasing
\let\tt\texttt
\let\bf\textbf
\let\it\itshape
\let\ul\underline

\title{Using the shell}
\date{26 October 2021}
\author{Johan Öhlin}
\institute{Yabs}

\begin{document}
% Page 1 (title) %%%%%%%%%%%%%%%%%%%%%%%%%%%%%%%%%%%%%%%%%%%%%%%%%%%%%%%%%%%%%
\maketitle

% Page 2 %%%%%%%%%%%%%%%%%%%%%%%%%%%%%%%%%%%%%%%%%%%%%%%%%%%%%%%%%%%%%%%%%%%%%
\begin{frame}{Me}
        Johan Öhlin           \\
        At Yabs since 2017    \\
        Primarily running GNU+Linux since 2008    \\
\end{frame}
% Page 3 %%%%%%%%%%%%%%%%%%%%%%%%%%%%%%%%%%%%%%%%%%%%%%%%%%%%%%%%%%%%%%%%%%%%%
\begin{frame}{The terminal}
        Started as a keyboard driven user interfeace to the operating system's
        kernel                                                            \\
        Grown to not primarily not communicate directly with the kernel   \\
        More `advanced' scripts to automate chores                        \\
\end{frame}

% Page 4 %%%%%%%%%%%%%%%%%%%%%%%%%%%%%%%%%%%%%%%%%%%%%%%%%%%%%%%%%%%%%%%%%%%%%
\begin{frame}{``Shell'' vs. ``terminal''}
        Pretty much used interchangeably                                  \\
        Terminal (emulator) controls few things and executes shell        \\
        Shell interprets `commands' and their interfaces                  \\
\end{frame}

% Page 5 %%%%%%%%%%%%%%%%%%%%%%%%%%%%%%%%%%%%%%%%%%%%%%%%%%%%%%%%%%%%%%%%%%%%%
\begin{frame}{Bash}
        Bourne-Again Shell (GNU, 1989)                                    \\
        Replacement of Bourne Shell (Bell, 1979)                          \\
        De-facto standard on GNU+Linux systems                            \\
        Available on Mac and Windows                                      \\
\end{frame}

% Page 6 %%%%%%%%%%%%%%%%%%%%%%%%%%%%%%%%%%%%%%%%%%%%%%%%%%%%%%%%%%%%%%%%%%%%%
\begin{frame}{Alteratives}
        \begin{tabular}{p{0.1\textwidth} p{0.4\textwidth} p{0.2\textwidth}}
                \tt{csh}        & C shell   & 1978--  \\
                \tt{fish}       & Friendly interactive shell &  2005-- \\
                \tt{ksh}        & Korn shell & 1983--2012 \\
                \tt{zsh}        & Z shell & 1990--  \\
        \end{tabular}
\end{frame}

% Page 7 %%%%%%%%%%%%%%%%%%%%%%%%%%%%%%%%%%%%%%%%%%%%%%%%%%%%%%%%%%%%%%%%%%%%%
\begin{frame}{File system}
        All directories has itself in it, called \tt{.}                \\
        All directories has previous directory in it, called \tt{..}   \\
        Your home directory is called \tt{~}                           \\
        Everything starts at the root directory, called \tt{/}         \\
        ``Everything is a file''. Eg. \tt{/dev/cdrom0}, \tt{/dev/hda0}, etc.
\end{frame}

% Page 8 %%%%%%%%%%%%%%%%%%%%%%%%%%%%%%%%%%%%%%%%%%%%%%%%%%%%%%%%%%%%%%%%%%%%%
\begin{frame}{Basics}
        \tt{\$ command [options and arguments]}  \\
        %\begin{tabular}{l l l}
        \begin{tabular}{p{0.1\textwidth} p{0.3\textwidth} p{0.3\textwidth}}
                \hline
                \tt{cd}           &
                Change directory  &
                \tt{\$ cd .././dir}  \\
                \hline
                \tt{rm}           &
                Remove &
                \tt{\$ rm "\$\{HOME\}/secrets.gz"}  \\
                \hline
                \tt{mkdir}           &
                Create directory &
                \tt{\$ mkdir newdir}  \\
                \hline
                \tt{rmdir}           &
                Remove \ul{empty} directory &
                \tt{\$ rmdir olddir}  \\
                \hline
                \tt{touch}               &
                Change file timestamps   &
                \tt{\$ touch newfile}       \\
                \hline
                \tt{mv}                     &
                Move a file (also rename)   &
                \tt{\$ mv olddir/file newdir}  \\
                \hline
        \end{tabular} \\
        Note that \tt{cd} is a built-in command and not an external program.
\end{frame}

% Page 9 %%%%%%%%%%%%%%%%%%%%%%%%%%%%%%%%%%%%%%%%%%%%%%%%%%%%%%%%%%%%%%%%%%%%%
\begin{frame}{Basics - Example}
        \tt{\$ mkdir "bash course"}         \\
        \tt{\$ mkdir "bash course/day 1"}   \\
        \tt{\$ cd "bash course/day 1"}      \\
        \tt{\$ cd ..}                       \\
        \tt{\$ rmdir "day 1"}               \\
        \tt{\$ touch "file 1" "file 2"}     \\
        \tt{\$ mv "file 2" "file 1"}        \\
        \tt{\$ rm "file 1"}                 \\
        \tt{\$ mv "file 1" "../file"}       \\
        \tt{\$ rm "../file"}                \\
\end{frame}

% Page 10 %%%%%%%%%%%%%%%%%%%%%%%%%%%%%%%%%%%%%%%%%%%%%%%%%%%%%%%%%%%%%%%%%%%%
\begin{frame}{How to}
        %\begin{tabular}{l l l}
        \begin{tabular}{p{0.1\textwidth} p{0.5\textwidth} p{0.3\textwidth}}
                \hline
                \tt{man}                    &
                Briefer \bf{man}ual page    &
                \tt{\$ man ls}              \\
                \hline
                \tt{info}                           &
                More extensivve \bf{info}rmation    &
                \tt{\$ info ls}                     \\
                \hline
                \tt{whatis}                         &
                Header of \tt{man} page.            &
                \tt{\$ whatis ls}                   \\
                \hline
                \tt{tldr}             &
                Web client for a project called tldr-pages; 
                \url{tldr.ostera.io} &
                \tt{\$ tldr ls}       \\
                \hline
        \end{tabular}   \\
        Online search engines.
\end{frame}

% Page 11 %%%%%%%%%%%%%%%%%%%%%%%%%%%%%%%%%%%%%%%%%%%%%%%%%%%%%%%%%%%%%%%%%%%%
\begin{frame}{File content}
        %\begin{tabular}{l l p{0.5 \textwidth}}
        \begin{tabular}{p{0.1\textwidth} p{0.4\textwidth} p{0.2\textwidth}}
                \hline
                \tt{cat}             &
                Concatenates and prints files on the standard output &
                \tt{\$ cat file1 file2}       \\
                \hline
                \tt{head}             &
                Outputs the first part of files &
                \tt{\$ head -n 20 file}       \\
                \hline
                \tt{tail}             &
                Outputs the last part of files &
                \tt{\$ tail -n 20 file}       \\
                \hline
                \tt{less}             &
                Outputs the first part of files &
                \tt{\$ head -n 20 file}       \\
                \hline
        \end{tabular}  
\end{frame}

% Page 12 %%%%%%%%%%%%%%%%%%%%%%%%%%%%%%%%%%%%%%%%%%%%%%%%%%%%%%%%%%%%%%%%%%%%
\begin{frame}{Finding programs}
        %\begin{tabular}{l l l}
        \begin{tabular}{p{0.2\textwidth} p{0.4\textwidth} p{0.2\textwidth}}
                \hline
                \tt{command -v}             &
                Finding location of program to run (from \$PATH) &
                \tt{\$ command -v ls}       \\
                \hline
                \tt{which}             &
                Finding location of program to run (from \$PATH) &
                \tt{\$ which ls}       \\
                \hline
        \end{tabular}
\end{frame}

% Page 13 %%%%%%%%%%%%%%%%%%%%%%%%%%%%%%%%%%%%%%%%%%%%%%%%%%%%%%%%%%%%%%%%%%%%
\begin{frame}{Core utils}
        A handful of GNU core utils 8,27. \\
        IEEE Std 1003.1-2008 utilities \\
        \begin{tabular}{p{0.1\textwidth} p{0.4\textwidth} p{0.3\textwidth}}
                \hline
                \tt{date}             &
                Prints or sets the system date and time &
                \tt{\$ date --date="fortnight"}       \\
                \hline
                \tt{dd}             &
                Copies and converts a file  &
                \tt{\$ dd if=/home/ubuntu.iso of=/dev/usb0}       \\
                \hline
                \tt{df}             &
                Shows disk free space on file systems   &
                \tt{\$ df -h dir/file.txt}       \\
                \hline
                \tt{du}             &
                Shows disk usage on file systems &
                \tt{\$ du -h dir/file.txt}       \\
                \hline
                \tt{ln}             &
                Creates a link (shortcut) to a file &
                \tt{\$ ln -s existing\_file link}       \\
                \hline
                \tt{false}             &
                Does nothing, but exits unsuccessfully &
                \tt{\$ false}       \\
                \hline
        \end{tabular}
\end{frame}

% Page 14 %%%%%%%%%%%%%%%%%%%%%%%%%%%%%%%%%%%%%%%%%%%%%%%%%%%%%%%%%%%%%%%%%%%%
\begin{frame}{Core utils}
        \begin{tabular}{p{0.1\textwidth} p{0.4\textwidth} p{0.3\textwidth}}
                \hline
                \tt{mktemp}             &
                Creates a temporary file or directory &
                \tt{\$ mktemp }       \\
                \hline
                \tt{nl}             &
                Numbers lines of files &
                \tt{\$ nl file}       \\
                \hline
                \tt{printf}             &
                Formats and prints data  &
                \tt{\$ printf "0x\%x" 20}       \\
                \hline
                \tt{pwd}             &
                Prints the current working directory &
                \tt{\$ pwd}       \\
                \hline
                \tt{seq}             &
                Prints a sequence of numbers  &
                \tt{\$ seq 1 7 32}       \\
                \hline
                \tt{shuf}             &
                Generate random permutations &
                \tt{\$ shuf -u fil}       \\
                \hline
                \tt{sleep}             &
                Delays for a specified amount of time &
                \tt{\$ sleep 30 1m}       \\
                \hline
        \end{tabular}
\end{frame}
% Page 15 %%%%%%%%%%%%%%%%%%%%%%%%%%%%%%%%%%%%%%%%%%%%%%%%%%%%%%%%%%%%%%%%%%%%
\begin{frame}{Core utils}
        \begin{tabular}{p{0.1\textwidth} p{0.4\textwidth} p{0.3\textwidth}}
                \hline
                \tt{sort}             &
                Sort lines of text files  &
                \tt{\$ sort -u file}       \\
                \hline
                \tt{stat}             &
                Returns data about an inode &
                \tt{\$ stat file}       \\
                \hline
                \tt{tac}             &
                Concatenates and prints files in reverse order line by line &
                \tt{\$ tac file}       \\
                \hline
                \tt{tee}             &
                Sends output to multiple files &
                \tt{\$ tee file}       \\
                \hline
                \tt{tr}             &
                Translates or deletes characters &
                \tt{\$ tr "{\textbackslash}n}" "{\textbackslash}t"       \\
                \hline
                \tt{true}             &
                Does nothing, but exits successfully &
                \tt{\$ true}       \\
                \hline
        \end{tabular}
\end{frame}

% Page 16 %%%%%%%%%%%%%%%%%%%%%%%%%%%%%%%%%%%%%%%%%%%%%%%%%%%%%%%%%%%%%%%%%%%%
\begin{frame}{Core utils}
        \begin{tabular}{p{0.1\textwidth} p{0.4\textwidth} p{0.3\textwidth}}
                \hline
                \tt{wc}             &
                Prints the number of bytes, words, and lines in files &
                \tt{\$ wc -l file}       \\
                \hline
                \tt{yes}             &
                Prints a string repeatedly &
                \tt{\$ yes no}       \\
                \hline
        \end{tabular}
\end{frame}

% Page 17 %%%%%%%%%%%%%%%%%%%%%%%%%%%%%%%%%%%%%%%%%%%%%%%%%%%%%%%%%%%%%%%%%%%%
\begin{frame}{Piping}
        Communication between programs.     \\
        \tt{\$ ls | wc -l | sort -u}        \\

        Writing to file.    \\
        \tt{\$ ls > file}   \\
        \tt{\$ > file ls}   \\

        Reading from file.    \\
        \tt{\$ wc < file}   \\
        \tt{\$ < file wc}
        
        Appending to file.    \\
        \tt{\$ echo hello >> file}   \\
        \tt{\$ >> file echo hello }
\end{frame}

% Page 18 %%%%%%%%%%%%%%%%%%%%%%%%%%%%%%%%%%%%%%%%%%%%%%%%%%%%%%%%%%%%%%%%%%%%
\begin{frame}{Piping}
        Combining execution.     \\
        \tt{\$ ls; echo "hello"}        \\

        First have to succeed.    \\
        \tt{\$ ls \&\& echo "hello"}        \\

        Execute second if first fails.    \\
        \tt{\$ ls || echo "hello"}        \\

        Combine them.    \\
        \tt{\$ (ls || echo "hello") \&\& pwd}        \\
\end{frame}
\end{document}
