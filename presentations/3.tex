\documentclass{beamer}

\usepackage[utf8]{inputenc}
\usepackage{hyperref}
\usetheme{metropolis}

% Aliasing
\let\tt\texttt
\let\bf\textbf
\let\it\itshape
\let\ul\underline
\let\tilde\texttildelow

\title{Using the shell}
\date{4 November 2021}
\author{Johan Öhlin}
\institute{Yabs}

\begin{document}
% Page 1 (title) %%%%%%%%%%%%%%%%%%%%%%%%%%%%%%%%%%%%%%%%%%%%%%%%%%%%%%%%%%%%%
\maketitle

% Page 2 %%%%%%%%%%%%%%%%%%%%%%%%%%%%%%%%%%%%%%%%%%%%%%%%%%%%%%%%%%%%%%%%%%%%%
\begin{frame}{Good to know}
        Commonly used keyboard shortcuts: \\
        \begin{itemize}
                \item Previously typed line: $\uparrow$ or C+p
                \item Other way around: $\downarrow$ or C+n
                \item Moving back and forward on line:
                        \begin{itemize}
                                \item Backward/forward one letter: $\leftarrow$ / $\rightarrow$ or C+b / C+f
                                \item Backward/forward one word: A+b / A+f
                                \item Start of line: C+a
                                \item End of line: C+e
                        \end{itemize}
                \item Search for used command with C+r
                \item Clear screen with C+l
                \item Execute with C+o or C+j
        \end{itemize}
\end{frame}

% Page 2.5 %%%%%%%%%%%%%%%%%%%%%%%%%%%%%%%%%%%%%%%%%%%%%%%%%%%%%%%%%%%%%%%%%%%
\begin{frame}{Good to know}
        Commonly used keyboard shortcuts: \\
        \begin{itemize}
                \item Deleting characters
                        \begin{itemize}
                                \item Remove previous letter: C+h
                                \item Remove next letter: C+d
                                \item Remove* previous word: C+w
                                \item Remove* next word: A+d
                                \item Remove* line: C+u
                                \item Remove* rest of line: C+k
                        \end{itemize}
                \item \bf{*Actually} they cut. Use C+y to paste what you just removed, but only the last keypress.
        \end{itemize}
\end{frame}

% Page 3 %%%%%%%%%%%%%%%%%%%%%%%%%%%%%%%%%%%%%%%%%%%%%%%%%%%%%%%%%%%%%%%%%%%%%
\begin{frame}{Good to know}
        Commonly used keyboard shortcuts: \\
        \begin{itemize}
                \item Search for used command with C+r (back) or C+s (forward)
                \item Clear screen with C+l
                \item Execute with C+o or C+j
                \item Cancel a command with C+c
                \item Terminate command or shell with C+d
        \end{itemize}
\end{frame}

% Page 4 %%%%%%%%%%%%%%%%%%%%%%%%%%%%%%%%%%%%%%%%%%%%%%%%%%%%%%%%%%%%%%%%%%%%%
\begin{frame}{Aliasing}
        \begin{itemize}
                \item You can make use of aliases if you are lazy and don't want to type as much. \\
                \item Make some guards for commands.    \\
                \item Make (short) scripts.             \\
                \item Make a \tt{command} out of your script.
        \end{itemize}
\end{frame}

% Page 5 %%%%%%%%%%%%%%%%%%%%%%%%%%%%%%%%%%%%%%%%%%%%%%%%%%%%%%%%%%%%%%%%%%%%%
\begin{frame}{Aliasing}
        To show your current aliases, only type \tt{alias}. \\
        \tt{\$ whatis alias} \\
        \tt{alias: nothing appropriate.} \\
        \bf{Note:} Different shells will have different ways of creating aliases. \\
        \begin{itemize}
                \item \tt{alias l='ls -Cf'}
                \item \tt{alias rm='rm -i'}
                \item \tt{alias ab='ls -p | wc -b; whoami'}
                \item \tt{alias xy='/path/to/script --with-flags'}
                \item \tt{alias listfiles='ls'}
                \item \tt{alias listfiles='l'}
        \end{itemize}
        See if you have any aliases already, and try to create new ones.
\end{frame}

% Page 6 %%%%%%%%%%%%%%%%%%%%%%%%%%%%%%%%%%%%%%%%%%%%%%%%%%%%%%%%%%%%%%%%%%%%%
\begin{frame}{Aliasing \& .bashrc}
        \begin{itemize}
                \item To make aliases permanent, i. e. be available each time you start a new instance, you can keep it a bash startup file.
                \item By default, you Bash startup file will be located in {\it{\tilde}/.bashrc}.
                \bf{Note:} Files starting with . are {\it hidden}, in the regard that \tt{ls} will not show them by default. You might have many various startup, configuration files and directories in your home directory. To see those files, you run \tt{ls -a}.
        \end{itemize}
\end{frame}

% Page 7 %%%%%%%%%%%%%%%%%%%%%%%%%%%%%%%%%%%%%%%%%%%%%%%%%%%%%%%%%%%%%%%%%%%%%
\begin{frame}{Aliasing \& .bashrc}
        \begin{itemize}
                \item The startup file will be loaded each time you start a new shell. If you wish to load this startup in an already opened shell, you can type \tt{source {\tilde}/.bashrc}.
                \item You will later learn about scripts and how to make functions. You startup may also contain similar functions which you can use as commands.
        \end{itemize}
        You can now try to edit your startup file. \\
        \bf{Cygwin users:} To locate your startup file, run \\
        \quad \tt{\$ realpath {\tilde}/.bashrc | xargs cygpath -w}
\end{frame}

% Page 8 %%%%%%%%%%%%%%%%%%%%%%%%%%%%%%%%%%%%%%%%%%%%%%%%%%%%%%%%%%%%%%%%%%%%%
\begin{frame}{.bashrc \& variables}
        Other than aliases and functions, your startup file can set variables which you might find useful. \\
        These can be set in the shell straight, but will be no kept between sessions. \\
        \tt{\$ HW="Hello world"}    \\
        \tt{\$ echo \$HW}           \\
        Your shell has plenty of these so-called environment variables already. Those can be viewed with \tt{env}   \\
        To make environment variables you will have the variables {\it exported}. \\
        To export, use command \tt{export} \\
        \tt{\$ export HW="Hello world"}    \\
\end{frame}

% Page 8.5 %%%%%%%%%%%%%%%%%%%%%%%%%%%%%%%%%%%%%%%%%%%%%%%%%%%%%%%%%%%%%%%%%%%
\begin{frame}{.bashrc \& variables}
        \tt{\$ echo \$HW }          \\
        \tt{ }                      \\
        \tt{\$ HW="Hello World" }   \\
        \tt{\$ echo \$HW }          \\
        \tt{Hello World}            \\
        \tt{\$ env | grep HW}       \\
        \tt{\$ export HW}           \\
        \tt{\$ env | grep HW}       \\
        \tt{HW=}                    \\
        \tt{\$ export HW="Hello World"} \\
        \tt{\$ env | grep HW}       \\
        \tt{HW=Hello World}         \\
\end{frame}

% Page 9 %%%%%%%%%%%%%%%%%%%%%%%%%%%%%%%%%%%%%%%%%%%%%%%%%%%%%%%%%%%%%%%%%%%%%
\begin{frame}{.bashrc \& variables}
        Variables within variables: \\
        \tt{\$ A="Hello"} \\
        \tt{\$ b="World"} \\
        \tt{\$ C="\$A \$b"} \\
        \tt{\$ C="\$C \$C"} \\
        \tt{\$ echo \$C} \\
        \tt{Hello World Hello World} \\
        Can be good practice to use (e. g.) \tt{\$HOME} variable instead of hard-coded path to {\it your} home directory. \\
        \bf{Note:} If you have a program which looks at the environment variables, you can override it as \tt{\$ VAR=yes env}
\end{frame}

% Page 10 %%%%%%%%%%%%%%%%%%%%%%%%%%%%%%%%%%%%%%%%%%%%%%%%%%%%%%%%%%%%%%%%%%%%
\begin{frame}{.bashrc \& functions}
        Result of command in variable.  \\
        \tt{\$ A=\$(ls) }               \\
        \tt{\$ echo \$A }               \\
        \tt{file1}                      \\
        \tt{file2}                      \\
        \tt{\$ echo \$A | head -n1 }    \\
        \tt{file1}                      \\
        \tt{\$}
\end{frame}

% Page 11 %%%%%%%%%%%%%%%%%%%%%%%%%%%%%%%%%%%%%%%%%%%%%%%%%%%%%%%%%%%%%%%%%%%%
\begin{frame}{.bashrc \& functions}
        You can have functions in your startup file instead of aliases. \\
        \tt{\$ cat {\tilde}/.bashrc}    \\
        \tt{function notls \{ }         \\
        \quad \tt{echo "This is not \$VAR"} \\
        \quad \tt{\$VAR | cat -n }      \\
        \tt{\} }                        \\
        \tt{VAR="ls"}                   \\
        \bf{Note:} This (above) can written directly into your shell. \\
\end{frame}

% Page 12 %%%%%%%%%%%%%%%%%%%%%%%%%%%%%%%%%%%%%%%%%%%%%%%%%%%%%%%%%%%%%%%%%%%%
\begin{frame}{Functions}
        Write in to your shell. The indentations are not important. \\
        \tt{\$ function notls \{ }         \\
        \quad \tt{echo "This is not \$VAR"} \\
        \quad \tt{\$VAR | cat -n }      \\
        \tt{\} }                        \\
        Run \tt{notls} in shell. Set \tt{VAR} and run again: \\
        \tt{\$ VAR="ls"}                   \\
        Change \tt{VAR} and run again:     \\
        \tt{\$ VAR="whoami"}               \\
\end{frame}

% Page 12 %%%%%%%%%%%%%%%%%%%%%%%%%%%%%%%%%%%%%%%%%%%%%%%%%%%%%%%%%%%%%%%%%%%%
\begin{frame}{Functions}
        You can call fucntions from functions.  \\
        \tt{\$ function hightlightk \{ }        \\
        \quad \# Colour the letter k            \\
        \quad \tt{notls | grep --color 'k'}     \\
        \tt{\} }                        \\
\end{frame}

% Page 13 %%%%%%%%%%%%%%%%%%%%%%%%%%%%%%%%%%%%%%%%%%%%%%%%%%%%%%%%%%%%%%%%%%%%
\begin{frame}{.bashrc \& functions}
        To use arguments:                       \\
        \tt{\$ function hightlightk \{ }        \\
        \quad \# Colour \$1 from notls          \\
        \quad \tt{notls | grep --color '\$1' }  \\
        \tt{\} }                                \\
\end{frame}

% Page 14 %%%%%%%%%%%%%%%%%%%%%%%%%%%%%%%%%%%%%%%%%%%%%%%%%%%%%%%%%%%%%%%%%%%%
\begin{frame}{functions}
        \bf{Warining:} You can recursively run you function. Just make sure to avoid {\it fork bombs}.  \\
        \tt{\$ function f \{ }        \\
        \quad \tt{ f | f {\&}}  \\
        \tt{\} }                                \\
\end{frame}

% Page 15 %%%%%%%%%%%%%%%%%%%%%%%%%%%%%%%%%%%%%%%%%%%%%%%%%%%%%%%%%%%%%%%%%%%%
\begin{frame}{\&}
        To start a process in the background, you can end the line with one single ampersand.   \\
        \tt{\$ sleep 30 {\&} }    \\
        To see which processes you have started, use the command \tt{jobs}. \\
        \tt{\$ jobs }  \\
        \tt{[2]- Kör sleep 20 {\&}}  \\
        You can set it as a foreground proccess again with the command \tt{fg}.
        \tt{\$ fg}  \\
        \tt{} \\
\end{frame}

% Page 16 %%%%%%%%%%%%%%%%%%%%%%%%%%%%%%%%%%%%%%%%%%%%%%%%%%%%%%%%%%%%%%%%%%%%
\begin{frame}{\&}
        To make a foreground process into a background process, you have to stop the process. This is achieved with C-z. \\
        You can later make it into a background process with the command \tt{bg}.   \\
        The process will still be connected to you shell. To dispatch it, use \tt{disown}. \\
        \tt{\$ jobs}                    \\
        \tt{\$ sleep 20}                \\
        \tt{\^Z}                        \\
        \tt{[1]+  Stoppad   sleep 20}   \\
        \tt{\$ bg}                      \\
        \tt{[1]+ sleep 20 {\&}}         \\
        \tt{\$ \# Wait for 20 seconds }  \\
        \tt{[1]+  Klart     sleep 20}   \\
        \tt{\$ }                        \\
\end{frame}

\end{document}
