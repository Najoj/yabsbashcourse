\documentclass{beamer}

\usepackage[utf8]{inputenc}
\usepackage{hyperref}
\usetheme{metropolis}

% Aliasing
\let\tt\texttt
\let\bf\textbf
\let\it\itshape
\let\ul\underline
\let\tilde\texttildelow

\title{Using the shell}
\date{4 November 2021}
\author{Johan Öhlin}
\institute{Yabs}

\begin{document}
% Page 1 (title) %%%%%%%%%%%%%%%%%%%%%%%%%%%%%%%%%%%%%%%%%%%%%%%%%%%%%%%%%%%%%
\maketitle

% Page 2 %%%%%%%%%%%%%%%%%%%%%%%%%%%%%%%%%%%%%%%%%%%%%%%%%%%%%%%%%%%%%%%%%%%%%
\begin{frame}{Good to know}
        Commonly used keyboard shortcuts: \\
        \begin{itemize}
                \item Previously typed line: $\uparrow$ or C+p
                \item Other way around: $\downarrow$ or C+n
                \item Moving back and forward on line:
                        \begin{itemize}
                                \item Backword/forward one letter: $\leftarrow$ / $\rightarrow$ or C+b / C+f
                                \item Backward/forward one word: A+b / A+f
                                \item Start of line: C+a
                                \item End of line: C+e
                        \end{itemize}
                \item Deleting charachters
                        \begin{itemize}
                                \item Remove previous letter: C+h
                                \item Remove next letter: C+d
                                \item Remove previous word: C+w
                                \item Remove next word: A+d
                                \item Clear line: C+u
                                \item Clear rest of line: C+k
                        \end{itemize}
                \item \it{\bf{Actually}} they cut. Use C+y to paste what you just removed
                \item Search for used command with C+r
                \item Clear screen with C+l
                \item Execute with C+o or C+j
        \end{itemize}
\end{frame}

% Page 3 %%%%%%%%%%%%%%%%%%%%%%%%%%%%%%%%%%%%%%%%%%%%%%%%%%%%%%%%%%%%%%%%%%%%%
\begin{frame}{Good to know}
        Commonly used keyboard shortcuts: \\
        \begin{itemize}
                \item Search for used command with C+r (back) or C+s (forward)
                \item Clear screen with C+l
                \item Execute with C+o or C+j
                \item Cancel a command with C+c
                \item Terminate command or shell with C+d
        \end{itemize}
\end{frame}

% Page 4 %%%%%%%%%%%%%%%%%%%%%%%%%%%%%%%%%%%%%%%%%%%%%%%%%%%%%%%%%%%%%%%%%%%%%
\begin{frame}{Aliasing}
        \begin{itemize}
                \item You can make use of aliases if you are lazy and don't want to type as much. \\
                \item Make some guards for commands.    \\
                \item Make (short) scripts.             \\
                \item Make a \tt{command} out of your script.
        \end{itemize}
\end{frame}

% Page 5 %%%%%%%%%%%%%%%%%%%%%%%%%%%%%%%%%%%%%%%%%%%%%%%%%%%%%%%%%%%%%%%%%%%%%
\begin{frame}{Aliasing}
        Command \tt{alias}. \\
        To show your curent aliases, only type \tt{alias}. \\
        \tt{\$ whatis alias} \\
        \tt{alias: nothing appropiate.} \\
        Different shells will have different ways of creating aliases. \\
        \begin{itemize}
                \item \tt{alias l='ls -Cf'}
                \item \tt{alias rm='rm -i'}
                \item \tt{alias ab='ls -p | wc -b; whoami'}
                \item \tt{alias xy='/path/to/script --with-flags'}
        \end{itemize}
        See if you have any aliases already, and try to create new ones.
\end{frame}

% Page 6 %%%%%%%%%%%%%%%%%%%%%%%%%%%%%%%%%%%%%%%%%%%%%%%%%%%%%%%%%%%%%%%%%%%%%
\begin{frame}{Aliasing \& .bashrc}
        \begin{itemize}
                \item To make aliases permanent, i. e. be available each time you start a new instance, you can keep it a bash startup file.
                \item By default, you Bash startup file will be located in {\it{\tilde}/.bashrc}.
                \item \bf{Note:} Files starting with . are {\it hidden}, in the regard that \tt{ls} will not show them by default. You might have many various startup, configuration files, and directories. To see those files, you run \tt{ls -a}.
        \end{itemize}
\end{frame}

% Page 7 %%%%%%%%%%%%%%%%%%%%%%%%%%%%%%%%%%%%%%%%%%%%%%%%%%%%%%%%%%%%%%%%%%%%%
\begin{frame}{Aliasing \& .bashrc}
        \begin{itemize}
                \item The startup file will be loaded each time you start a new shell. If you wish to load this startup in an already opened shell, you can type \tt{source {\tilde}/.bashrc}.
                \item You will later learn about scripts and how to make functions. You startup may also contain similar functions which you can use as commands.
        \end{itemize}
        You can now try to edit your startup file. \\
        \bf{Cygwin users:} To locate your startup file, run \tt{realpath {\tilde}/.bashrc | xargs cygpath -w}
\end{frame}

% Page 8 %%%%%%%%%%%%%%%%%%%%%%%%%%%%%%%%%%%%%%%%%%%%%%%%%%%%%%%%%%%%%%%%%%%%%
\begin{frame}{.bashrc \& variables}
        Other than aliases and fucntions, your startup file can set variables which you might find useful. \\
        These can be set in the shell straight, but will be no kept between sessions. \\
        \tt{\$ HW="hello world"}    \\
        \tt{\$ echo \$HW}           \\
        Your shell has plenty of these so-called environemnt variablpes already. Those can be viewed with \tt{env}   \\
        To make enviroment variables you will have the variables ``exported''. \\
        To export, use command \tt{export} \\
        \tt{\$ export HW="hello world"}    \\
        \tt{\$ echo \$HW}           \\
\end{frame}

% Page 9 %%%%%%%%%%%%%%%%%%%%%%%%%%%%%%%%%%%%%%%%%%%%%%%%%%%%%%%%%%%%%%%%%%%%%
\begin{frame}{.bashrc \& variables}
        Variables within variables: \\
        \tt{\$ A="Hello"} \\
        \tt{\$ B="World"} \\
        \tt{\$ C="\$A \$B"} \\
        \tt{\$ C="\$C \$C"} \\
        \tt{\$ echo \$C} \\
        \tt{Hello World Hello World} \\
        Can be good practice to use (e. g.) \tt{\$HOME} variable instead of hard-coded path to {\it your} home directory. \\
        \bf{Note:} If you have a program which looks at the environment variables, you can override it as \tt{\$ VAR=yes env}
\end{frame}

% Page 10 %%%%%%%%%%%%%%%%%%%%%%%%%%%%%%%%%%%%%%%%%%%%%%%%%%%%%%%%%%%%%%%%%%%%
\begin{frame}{.bashrc \& variables}
\end{frame}
\end{document}
