\documentclass{beamer}

\usepackage[utf8]{inputenc}
\usepackage{hyperref}
\usetheme{metropolis}

% Aliasing
\let\tt\texttt
\let\bf\textbf
\let\it\itshape
\let\ul\underline

\title{Using the shell}
\date{11 November 2021}
\author{Johan Öhlin}
\institute{Yabs}

\begin{document}

% Page 1 (title) %%%%%%%%%%%%%%%%%%%%%%%%%%%%%%%%%%%%%%%%%%%%%%%%%%%%%%%%%%%%%
\begin{frame}{Using the shell}
\maketitle
\end{frame}

% Page 1.5 %%%%%%%%%%%%%%%%%%%%%%%%%%%%%%%%%%%%%%%%%%%%%%%%%%%%%%%%%%%%%%%%%%%
\begin{frame}{Debugging files}
    Files for exercises should be available at:    \\
    \url{https://github.com/Najoj/yabsbashcourse/tree/main/exercises/5} \\
    \bf{Note:} The notes will tell you which file to try, but you are free to use whichever you want. Therefore, only \tt{script.bash} will be used for the examples.
\end{frame}

% Page 1.75 %%%%%%%%%%%%%%%%%%%%%%%%%%%%%%%%%%%%%%%%%%%%%%%%%%%%%%%%%%%%%%%%%%
\begin{frame}{Debugging}
    Check syntax without running with \tt{-n}.
\end{frame}

% Page 2 %%%%%%%%%%%%%%%%%%%%%%%%%%%%%%%%%%%%%%%%%%%%%%%%%%%%%%%%%%%%%%%%%%%%%
\begin{frame}{Debugging}
    Print every line executed:      \\
    \tt{\$ bash -x script.bash}     \\
    or, set \tt{-x} argument in script head \\
    \tt{\#!/bin/bash -x}            \\
    or, set with \tt{set} command inside script \\
    \tt{\#!/bin/bash}               \\
    \tt{set -x  \# Note the minus sign} \\
    \tt{\# ...}                         \\
    \tt{set +x  \# Note the plus sign}  \\
\end{frame}

% Page 3 %%%%%%%%%%%%%%%%%%%%%%%%%%%%%%%%%%%%%%%%%%%%%%%%%%%%%%%%%%%%%%%%%%%%%
\begin{frame}{Debugging}
    Print shell input lines as they are read:   \\
    \tt{\$ bash -v script.bash}     \\
    or, set \tt{-v} argument in script head \\
    \tt{\#!/bin/bash -v}            \\
    or, set with \tt{set} command inside script \\
    \tt{\#!/bin/bash}               \\
    \tt{set -v  \# Note the minus sign}  \\
    \tt{\# ...}                          \\
    \tt{set +v  \# Note the plus sign}   \\
    \quad                           \\
    Can of course be combined with \tt{-x}.
\end{frame}

% Page 4 %%%%%%%%%%%%%%%%%%%%%%%%%%%%%%%%%%%%%%%%%%%%%%%%%%%%%%%%%%%%%%%%%%%%%
\begin{frame}{Debugging}
    Download \tt{script1.bash} from GitHub without and with commands above. \\
    Note what happens and how they differ.      \\
    Is the output expected?                     \\
\end{frame}

% Page 4 %%%%%%%%%%%%%%%%%%%%%%%%%%%%%%%%%%%%%%%%%%%%%%%%%%%%%%%%%%%%%%%%%%%%%
\begin{frame}{Debugging}
    Show full execution path with \tt{-ilx}.    \\
    \bf{Tip:} Long output can be stored in a file by using  \\
    \quad \tt{\$ bash -ilx script.bash \&> save.txt}         \\
    Or read through less
    \quad \tt{\$ bash -ilx script.bash 2>\&1 | less}         \\
    The execution steps are written to stderr. \tt{2>\&1} will redirect stderr to stdout. \tt{less} will only read streams from stdin and only data from stout will be directed to stdin.
\end{frame}

% Page 5 %%%%%%%%%%%%%%%%%%%%%%%%%%%%%%%%%%%%%%%%%%%%%%%%%%%%%%%%%%%%%%%%%%%%%
\begin{frame}{Debugging}
    Using debug variable \tt{DEBUG}.    \\
    \tt{\$ DEBUG= bash script.bash}      \\
    \quad   \\
    \tt{\#!/bin/bash}                   \\
    \tt{if \[ \$\{DEBUG+x\}; then \] }              \\
    \quad \tt{\# Set debug options an variables}    \\
    \tt{fi}                                         \\
    \bf{Note:} If parameter \tt{DEBUG} is null or unset, nothing is substituted, otherwise the expansion of word is substituted.   \\
\end{frame}

% Page 6 %%%%%%%%%%%%%%%%%%%%%%%%%%%%%%%%%%%%%%%%%%%%%%%%%%%%%%%%%%%%%%%%%%%%%
\begin{frame}{Debugging}
    Error printing function:                       \\
    \tt{function err \{ }                          \\
    \quad \tt{ [ "\$\{DEBUG+x\}" ] \&\& echo "\$@"}  \\
    \tt{\}}
\end{frame}

% Page 7 %%%%%%%%%%%%%%%%%%%%%%%%%%%%%%%%%%%%%%%%%%%%%%%%%%%%%%%%%%%%%%%%%%%%%
\begin{frame}{Shellcheck}
    \url{https://www.shellcheck.net/}               \\
    \quad       \\
    Shellcheck is a GPLv3 tool that gives warnings and suggestions for bash/sh shell scripts.   \\
    Can be installed or run through their website.      \\
    \quad   \\
    (Written in Haskell.)
\end{frame}

% Page 8 %%%%%%%%%%%%%%%%%%%%%%%%%%%%%%%%%%%%%%%%%%%%%%%%%%%%%%%%%%%%%%%%%%%%%
\begin{frame}{Shellcheck}
    Correct \tt{script1.bash} \& \tt{script2.bash} on Shellcheck. Have look at their explanations.
\end{frame}

% Page 9 %%%%%%%%%%%%%%%%%%%%%%%%%%%%%%%%%%%%%%%%%%%%%%%%%%%%%%%%%%%%%%%%%%%%%
\begin{frame}{Shellcheck}
    Errors in Shellcheck can be ignored by comment before error line:    \\
    \tt{\# shellcheck disable=SC2059}
\end{frame}

% Page 10 %%%%%%%%%%%%%%%%%%%%%%%%%%%%%%%%%%%%%%%%%%%%%%%%%%%%%%%%%%%%%%%%%%%%
\begin{frame}{Line-by-line execution}
    One way of forcing line-by-line execution of scripts is to use \tt{trap}:   \\
    \tt{\#!/bin/bash}       \\
    \tt{trap read debug}    \\
    \tt{whoami}             \\
    \tt{ls}                 \\
    \tt{for i in \{1..5\}; do echo "\$i"; done}                  \\
    \quad   \\
    Followed by your code. Press return to execute next line.   \\
    Try to run with \tt{-x} and \tt{-v} flag.
\end{frame}

% Page 11 %%%%%%%%%%%%%%%%%%%%%%%%%%%%%%%%%%%%%%%%%%%%%%%%%%%%%%%%%%%%%%%%%%%%
\begin{frame}{Various IDE plugins}
    You can search for plugins for your IDE. There will probably be some tool to help you.   \\
    \quad   \\
    There is a tool you can install to use \tt{--debug} flag for Bash called bashdb:    \\
    \url{http://bashdb.sourceforge.net/}
\end{frame}
\end{document}
